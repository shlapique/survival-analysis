\textbf{1. Место и сроки проведения практики}

\vspace{20pt}

{\em Дата начала практики \hspace{3cm} 29 \underline{июня} 2022 г.} \\

{\em Дата окончания практики \hspace{3cm} 12 \underline{июня} 2022 г.}

\vspace{20pt}

{\em Наименование предприятия \underline{МОСКОВСКИЙ АВИАЦИОННЫЙ ИНСТИТУТ(НИУ)}} \\

{\em Название структурного подраздления \underline{Кафедра 804}}

\vspace{20pt}

\textbf{2. Инструктаж по технике безопасности}

\vspace{10pt}

\underline{ Платонов Е. Н. / \hspace{3cm} /} \hspace{1cm} 29 \underline{июня} 2022 г.

\vspace{10pt}

\textbf{3. Индивидуальное задание студенту}

Изучение основной литературы по теме "Анализ выживаемости". Постановка задачи, область применения, основные определения и подходы к решению задачи. Пример решения задачи.

\vspace{10pt}
\pagebreak

\textbf{4. План выполнения индивидуального задания}

\begin{itemize}
    \item[1.] Изучить теорию по Анализу выживаемости.
    \item[2.] Ознакомиться с необходимыми библиотеками для работы с данными и их графическим представлением.
    \item[3.] Решить задачу по анализу данных с применением методов анализа выживаемости.
\end{itemize}

{\em Руководитель практики от МАИ: \underline{Зайцева О.Б. / \hspace{3cm} /}}

\vspace{20pt}

\underline{Платонов Е. Н. / \hspace{3cm} /} 29 \underline{июня} 2022 г.

\vspace{10pt}
\pagebreak

\textbf{5. Отзыв руководителя практики}

{\em Задание на практику выполнено в полном объеме. Материалы, изложенные в отчете студента, полностью соответствуют индвидуальному заданию. Рекомендую оценку отлично. }

\vspace{30pt}

{\em Руководитель \hspace{3cm} \underline{ Платонов Е. Н. / \hspace{3cm} /} 12 \underline{июля} 2022 г.}
\pagebreak
